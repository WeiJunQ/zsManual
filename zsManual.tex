\documentclass[a4paper,twoside,cs4size,fancyhdr,notitlepage]{ctexart}
% 页面布局

\usepackage[top=2.54cm,bottom=2.54cm,left=3.17cm,right=3.17cm]{geometry} % 用于设置页面布局


% 字体和段落

    % 章节标题显示方式
\CTEXsetup[format={\Large\raggedright\bfseries}]{section}

    % 字体设置
% \usepackage{xeCJK}  % 用于使用本地字体


    %设置常用中文字号,方便调用


% 内容布局

    % 页眉页脚页码
\pagestyle{fancy}
\fancyhf{}
\lhead{黎振胜}
\rhead{\leftmark}
\cfoot{\thepage}

    % 表格和图片
\usepackage{amssymb}
\usepackage{amsmath}    % 在公式中用\text{文本}输入中文
\usepackage{diagbox}    % 斜线表
\usepackage{multirow}   % 表格中使用多行
\usepackage{booktabs}   % 三线表
\usepackage{rotating}   % 使用sidewaystable环境旋转表格
\usepackage{tabularx}   % 定宽表格
\usepackage{graphicx}   % 处理图片
\usepackage{footnote}   % 增强的脚注功能,可添加表格脚注
\usepackage{threeparttable} % 添加真正的表格脚注,示例见README
\usepackage{hyperref}   % 添加pdf书签


% 目录和文献

\usepackage[super, square, sort&compress]{natbib}   % 处理参考文献

% 代码格式
\usepackage{listings}   % 添加代码高亮
\lstset{
	basicstyle=\footnotesize,       % the size of the fonts that are used for the code
	numbers=left,                   % where to put the line-numbers
	numberstyle=\footnotesize,      % the size of the fonts that are used for the line-numbers
	numberstyle=\tiny,
	stepnumber=5,                  	% the step between two line-numbers. If it is 1 each line will be numbered
	numbersep=5pt,                  % how far the line-numbers are from the code
	% showspaces=false,            	% show spaces adding particular underscores
	% showstringspaces=false,      	% underline spaces within strings
	showtabs=false,                 % show tabs within strings adding particular underscores
	% frame=single,           		% adds a frame around the code
	tabsize=1,          			% sets default tabsize to 2 spaces
	breaklines=true,       			% sets automatic line breaking
	postbreak=\raisebox{0ex}[0ex][0ex]{\ensuremath{\color{red}\hookrightarrow\space}},
	showlines=true
}


% 命令重定义 
% \renewcommand{\refname}{\bfseries{参~考~文~献}} 	%将Reference改为参考文献(用于 article)
\renewcommand{\bibname}{参~考~文~献}	%将bibiography改为参考文献(用于 book)
\renewcommand{\baselinestretch}{1.38} 	%设置行间距
\renewcommand{\figurename}{\small\ttfamily 图}
\renewcommand{\tablename}{\small\ttfamily 表}

\title{P3AT机器人实验手册}
\author{黎振胜}
\date{\today}

\begin{document}
\maketitle{}

\section{必读}
大家好,这是我离校之前对使用P3AT机器人的一些方法介绍和总结,不全面但都是一些很基本和重要的东西,使用机器人之前把这个浏览一遍可以少走弯路,希望对大家以后的实验有帮助。

另外这算是我第二次使用\LaTeX 进行写作,权当作一次练习,希望大家以后也多多尝试。

最后,希望后面的师弟能够把这份文档维护下去,比如加入图片。既可以学习\LaTeX ,又可以学习Git,何乐而不为。

\subsection{基本使用}
机器人为了正常连接激光设备,需要为激光分配固定IP地址,配置如下。如果需要电脑上网,则需要将网卡配置为自动获取IP。
这使用起来比较难受,但是目前没有找到解决办法。
\begin{lstlisting}
# set ip config in network manager
# Address	192.168.0.4
# Netmask	192.0.255.255(automatically become 192.0.0.0, could work but I don't know why)
# Gateway	192.168.0.1
\end{lstlisting}
\subsubsection{硬件}
灯提示及声音提示
\subsubsection{软件}
自己看readme安装也行,看后面的记录也行。
\subsection{充电}
\subsubsection{原装电池}
\subsubsection{高性能电池}
\subsection{里程计校准}
\subsection{资料清单}
\subsection{注意事项}
重新整理后,分文件夹进行介绍
\section{基于ROS的P3AT控制软件}
命令和代码仅为示例,请在了解Linux基本原理和工具的基础上阅读使用。
\lstset{
    language=bash % 第二章的代码默认是bash
}
\subsection{SDK和工具}
\subsubsection{libaria}
aria是Adept为P3AT机器人提供的核心开发包,能够屏蔽硬件细节,管理激光,声纳,摄像头等附件,此节介绍如何安装和使用这个SDK
\subparagraph{安装}
直接用dpkg命令安装.deb包,这里安装的是libaria2.9.0,命令如下
\begin{lstlisting}[language=bash]
sudo dpkg -i libaria_2.9.0+ubuntu12+gcc4.6_amd64.deb
sudo apt -f install
\end{lstlisting}
\subparagraph{查看手册}
SDK安装后到其目录下\path{/usr/local/aria}可以看到软件开发手册,点进去仔细阅读,尤其是开头的部分,如下图所示,应全部阅读。
\subparagraph{运行示例程序}
在\path{.../examples}目录下有一个cpp项目,使用make构建,包含基本的开发示例,开发文档中对这些文件进行了介绍。按以下命令编译运行示例程序。
\begin{lstlisting}[language=bash]
cd /usr/local/aria/examples
make simplemotioncommands
./simplemotioncommands
\end{lstlisting}
\subsubsection{MobileSim}
MobileSim是一款基于Stage的仿真器,使用这个工具可以从某种程度上代替真正物理机,方便进行软件调试,注意这个工具是二维的,功能有限
如果需要模拟更多传感器和执行器,推荐使用Gazebo。
\subparagraph{安装}
也是安装deb包,命令如下
\begin{lstlisting}
sudo dpkg -i mobilesim_0.7.3+ubuntu12+gcc4.6_amd64.deb
sudo apt -f install
\end{lstlisting}
\subparagraph{使用}
在MobileSim安装目录下的rademe中可以看到详细的使用方法,请浏览阅读。为了方便使用,在.bashrc中配置环境变量,示例命令如下
\begin{lstlisting}
# in bashrc
alias MobileSim="/usr/local/MobileSim/MobileSim"
# in shell
MobileSim -m cmee.map -r p3at
\end{lstlisting}
\subsection{配置zsROS}
\subsubsection{安装ROS}
到wiki页面查询安装方法,在14.04Ubuntu上安装Indigo版本\footnote{http://wiki.ros.org/indigo/Installation/Ubuntu}
建立工作空间。全部按照默认方法建立catkin\_ws\footnote{http://wiki.ros.org/ROS/Tutorials/InstallingandConfiguringROSEnvironment},配置.bashrc环境变量,示例如下:
\begin{lstlisting}[language=bash]
source /opt/ros/indigo/setup.bash
export EDITOR='subl'
export ROS_WORKSPACE=/home/zs/catkin_ws
export ROS_PACKAGE_PATH=${ROS_WORKSPACE}:${ROS_PACKAGE_PATH}
source ${ROS_WORKSPACE}/devel/setup.bash
alias MobileSim="/usr/local/MobileSim/MobileSim"
\end{lstlisting}
subsubsection{安装附件ROS包}
安装一些没有预装的ROS包,命令如下
\begin{lstlisting}
sudo apt install libsdl1.2-dev libsdl-image1.2-dev ros-indigo-navigation ros-indigo-axis-camera ros-indigo-flir-ptu-driver ros-indigo-serial
sudo apt install ros-indigo-freenect-stack ros-indigo-openni* libfreenect-dev
sudo apt install ros-indigo-joy-listener ros-indigo-joy-teleop ros-indigo-teleop-tools ros-indigo-teleop-twist-joy ros-indigo-teleop-twist-keyboard
sudo apt install ros-indigo-lms1xx
sudo apt install ros-indigo-rosbridge-suite
\end{lstlisting}
\subsubsection{编译zsROS包}
这个包是我整理设计的,就在github下存放着,注意下载link-withmap分支的代码,这个代码是最新的,编译方法如下。
\begin{lstlisting}
cd ./catkin_ws
cakkin_make
# 如果出错,则按下列步骤重新编译
cakkin_make
cd ./build
make
# 注意不要使用多线程编译,使用单线程就好,我遇到过这种错误,原因未知
\end{lstlisting}
\subsection{手柄控制}
此处讲述如何在本地使用手柄控制机器人,此处的机器人既可以指MobileSim模拟器,也可以指真机,
因为都是运行在Ubuntu上,所以原理一致。
\subparagraph{连接XBOX手柄}
手柄插入USB接口,在/dev/input中查看手柄对应的设备ID,如js0
\subparagraph{必要时修改launch文件}
根据对应的手柄设备ID,修改\path{E:\Zhensheng\documents\GitHub\zsROS\zs_launch\launch\zs_joy_rob.launch}文件。
如果对其他参数有兴趣也可修改,launch文件全文如下
\begin{lstlisting}[language=xml]
<launch>
  <node pkg = "joy" type = "joy_node" name = "joy_node" output="screen">
    <param name="dev" 			    value="/dev/input/js0" type="string"/>
    <param name="deadzone" 		    value="0.05"/>
    <param name="autorepeat_rate"   value="0.0"/>
    <param name="coalesce_interval" value="0.001"/>
  </node>
  <node pkg = "teleop_twist_joy" type = "teleop_node" name = "teleop_node" output="screen">
    <param name="enable_button"         value="0"/>
    <param name="enable_turbo_button"   value="-1"/>
    <param name="axis_linear"           value="1"/>
    <param name="scale_linear_turbo"    value="1.0"/>
    <param name="axis_angular"          value="0"/>
    <param name="scale_angular"         value="1.0"/>
    <remap from="cmd_vel" to="RosAria/cmd_vel" />
  </node>
</launch>
\end{lstlisting}
\subparagraph{启动ROS系统}
依次启动调试工具,ROSARIA传感驱动节点,RViz可视化工具,最后启动手柄驱动节点。命令示例如下
\begin{lstlisting}
# 启动ros master
roscore
# 启动调试工具
roscd zs_launch_files/
./zs_tools.sh cmee
# 启动ROSARIA
roslaunch zs_launch_files zs_rosaria.launch
# 启动RViz
roslaunch zs_launch_files zs_p3at_rviz.launch
# 启动手柄驱动结点
roslaunch zs_launch_files zs_joy_rob.launch
\end{lstlisting}
\subsection{记录里程计信息}
这里介绍使用rosbag记录里程计信息
\subparagraph{rosbag基本使用}
bag实际上是将一个node的行为录制下来,然后可以重新播放;
使用命令行能够:录制,从包重新发布,获取包的概括信息,检查包的消息类型,使用Python表达式过滤包中信息,压缩解压缩包,重新索引包;
如果录制高带宽的东西,例如图像信息,建议在本地录制和存放文件;
bag文件默认名字为日期,可以添加前缀。
\subparagraph{录制信息}
简单讲,就是指定主题进行录制,命令示例如下
\begin{lstlisting}
rosbag record rosout tf cmd_vel
rosbag record -a #所有
rosbag record --duration=30 /chatter #持续30s,5m,2h
rosbag record --split --size=1024 /chatter #空间达到1024M后分文件存储
rosbag record --split --duration=30 /chatter #持续时间到30s后分文件存储
rosbag record -o session1 /chatter #为文件名字做前缀
rosbag record -O session2_090210.bag /chatter #为文件命名
rosbag record --node=/joy_teleop #录制该节点左右消息
rosbag record -l 1000 /chatter  # 录制该主题1000个消息限制
\end{lstlisting}
\subparagraph{重放}
如果两个包被播放,它们将作为一个包对待。比如你录制一个包,在一个小时之后再录制一个包,然后同时播放,这个时候你会经历1个小时的空白。
示例命令如下
\begin{lstlisting}
rosbag play recorede1.bag
rosbag play --clock recorded1.bag #发布时钟时间
rosbag play --clock --hz=200 recorded1.bag #时钟频率默认100
rosbag play -r 10 recorded1.bag # 发布速度翻倍
rosbag play -d 5 recorded1.bag #每一次advertise后等待5秒
rosbag play -u 240 recorded1.bag #只播放240s
rosbag play -l recorded1.bag # 循环播放
\end{lstlisting}
\subparagraph{使用MATLAB进一步处理}
方法是先导出为CSV文件,再存为文件。示例命令如下
\begin{lstlisting}
# 先做实验显示
rostopic echo -b log_file.bag /topic_name
rostopic echo -p /topic_name #使用matlab友好格式进行显示
# 这个命令就可以了
rostopic echo -b file.bag -p /topic > data.txt
\end{lstlisting}
\subparagraph{注意}
如果图方便,可以使用rqt\_bag,这是一个基于qt的图形界面工具
\subsection{使用kinect}
这个很复杂,基本思路是使用\href{https://github.com/OpenKinect/libfreenect2}{libfreenect2这个库},
在这个库的基础上使用\href{https://github.com/code-iai/iai_kinect2}{iai-kinect2这个ROS包},但是库包并不受官方支持,所以很不稳定,必须根据每台计算机不断调试才能成功。
以下仅为示例代码,本人电脑为AMD显卡。
\begin{lstlisting}
# 安装libfreenect2
cd libfreenect2
cd depends; ./download_debs_trusty.sh
sudo apt-get install libusb-1.0-0-dev
sudo apt-get install libturbojpeg libjpeg-turbo8-dev
sudo apt-get install libglfw3-dev
sudo apt-get install beignet-dev
sudo apt-get install libva-dev libjpeg-dev
sudo apt-get install libopenni2-dev
cd ..
mkdir build
cd ./build
cmake ..  -DENABLE_CXX11=ON
make -j4
sudo make install
# 安装在 /usr/local/
# 测试库文件的有效性
# 最后可以运行程序,在build下面有个bin文件夹,放置生成的输出文件,插上kinect,然后运行。此时黄灯变成白色的,表示有驱动。注意:只能用于USB3的接口,好在台式机和笔记本都有3.0的口。改成如下的超级命令即可。 
sudo ./bin/Protonect
./Protonect cl 查看依赖错误信息
# 复制iai-kinect2ROS包到catkin_ws
catkin_make
# 测试ROS包正常工作
roslaunch kinect2_bridge kinect2_bridge.launch
\end{lstlisting}
\subsection{与远程人机交互平台配合使用}
\label{subsec:use-with-labview}
\subparagraph{修改bashrc文件}
根据具体情况,添加下位机(本机)IP地址,设置环境变量,示例如下。注意,如果不是作为下位机运行ROS,则环境变量得改回来。
\begin{lstlisting}
# 假设下位机IP地址为10.0.126.3
export ROS_IP=10.0.126.3
export ROS_MASTER_URI=http://10.0.126.3:11311
\end{lstlisting}
\subparagraph{启动ROS系统}
如果是仿真的话,则直接启动即可,如果机器人已经在远程运行,则需要使用SSH等远程登陆工具运行。示例命令如下
\begin{lstlisting}
# 启动ros master
roscore
# 启动调试工具(仿真环境下启动,真机条件下无需启动)
roscd zs_launch_files/
./zs_tools.sh cmee
# 启动ROSARIA(真机条件下需要打开串口权限)
# sudo chmod a+rwx /dev/ttyS0
roslaunch zs_launch_files zs_rosaria.launch
# 启动代理结点
roslaunch zs_launch_files zs_proxy.launch
# 启动导航避障模块
roslaunch zs_launch_files zs_move_base.launch
# 在必要时启动RViz可视化模块
# roslaunch zs_launch_files zs_p3at_rviz_ex.launch
\end{lstlisting}
\section{使用远程人机交互平台}
已经把所有软件安装到老板笔记本电脑上,下面是我的记录
\subsection{软件安装}
你们看到这个文档的时候环境应该已经配置好了,但我把它记下来以备不时之需。
\subsubsection{安装LabVIEW2014}
\subparagraph{安装软件}
安装软件,大家都会装吧,只需注意一点,安装所有软件功能,咱们实验室的LabVIEW包含功能特别多,全部装上去,安装过程如下
\subparagraph{安装更新}
使用前最好彻底更新,方法是【帮助】-【检查更新】。网络条件不好,记住一个一个装。
\subparagraph{升级VI Package Manager}
升级到最新版本,当前是2016,目录为.../软件依赖项/labview2014/vipm-windows.exe
\subsubsection{安装LabVIEW附加包}
使用VI Package Manager进行安装,可以在线安装,也可离线安装,文件夹中已经有离线包。
\subparagraph{lvh\_toolbox-2.0.0.35.vip}
这个是安装lvh\_xbone基础包,直接安装即可
\subparagraph{lvh\_xbone-2.0.0.6.vip}
这个是LabVIEW中的xbox手柄驱动程序,安装之后学习示例程序就可以了
\subparagraph{ni\_lib\_state\_pattern\_actor-1.1.0.10.vip}
这个是操作者框架的状态模式编程工具包,由于LabVIEW2014自带Actor Framework,所以可直接安装。
安装目录在\path{D:\Program Files (x86)\National Instruments\LabVIEW 2014\vi.lib\ActorFramework\}。
\footnote{见\url{https://forums.ni.com/t5/Actor-Framework-Discussions/Implementing-the-State-Pattern-in-Actor-Framework/td-p/3409456?tstart=0}}
\subparagraph{ROS for LabVIEW}
这个是Tufts University的师生制作的LabVIEW与ROS通信工具,其基于ROS底层通信协议(xml-RPC),可以在LabVIEW端启动roscore,
新建节点,发送订阅消息等。现在GitHub上开源,但是早已停止维护,其功能在新版本的ROS中部分无法使用。安装方法是将文件夹
复制到\path{D:\Program Files (x86)\National Instruments\LabVIEW 2014\vi.lib\}下。
\subsubsection{安装自然语言处理运行环境}
\subparagraph{Anaconda2-2.5.0-Windows-x86\_64.exe}
这是Python运行环境
\subparagraph{jdk-8u31-windows-i586.exe}
java运行环境,用于运行基于科大讯飞的语音识别窗口程序
\subparagraph{aiml-0.8.6}
与语音生成相关的包,使用python setup.py方法安装
\subparagraph{PyAIML-master}
AIML的Python接口
\subparagraph{pymdptoolbox-master}
马尔科夫决策过程工具箱的Python接口,与对话管理器的状态机有关
\subsubsection{安装语义推理运行环境}
\subparagraph{swipl-w64-723.exe}
swi-prolog是最受欢迎的Prolog开发环境,安装本环境用于运行基于语义的路径规划,安装完成记得添加PATH环境变量
\subparagraph{pyswip-0.2.3}
这是swip的Python接口包在线安装,也可离线安装,文件夹中已经有离线包。
\subsection{软件使用}
LabVIEW运行的是上位机软件,通过与ROS通信,达到控制P3AT移动机器人的目的,由于ROS为服务端,所以务必先启动P3AT ROS系统。
需要注意的是,ROS同样也有一套自己的控制界面系统rqt,熟悉ROS的同学敬请把这些移植到Ubuntu上。这一套是我基于LabVIEW实现的。
\subparagraph{运行ROS系统}
按照如\ref{subsec:use-with-labview}所述方法启动下位机ROS系统
\subparagraph{打开上位机软件}
进入\path{E:\Zhensheng\documents\GitHub\zsLV},打开zsLV.lvproj,进入LabVIEW项目管理器,打开\path{\LaunchCOUI.vi},
在控件中输入下位机IP地址,启动vi,会弹出控制面板和地图两个用户界面,当控制面板上接收到机器人的电压值时,说明连接成功,
用户也可以查看控制面板上的消息日志记录来查看软件使用情况。
\subparagraph{注意}
在使用上位机的同时,同样可以使用ROS的RViz工具对ROS系统进行监控以弥补上位机对数据可视化的不足,上位机程序主要目的是验证自然语言处理
以及语义规划算法的使用效果。
\appendix

\section{LabVIEW基础}
\subsection{LabVIEW调用其他程序}
\subsection{LabVIEW Actor Framework}
\subsection{LabVIEW与ROS}
\section{Linux使用和开发基础}
\section{Git基础}
\section{ROS基础}


\end{document}

